\chapter{Технологический раздел}
\section{Требования к программному обеспечению}
Программа должна отвечать следующим требованиям:
\begin{itemize}
	\item на вход программе подаются массив целых чисел и целое число, которое будет искаться в нём;
	\item осуществляется выбор алгоритма на исследование;
	\item на вход программе подаются только корректные данные;
	\item в зависимости от выбранного пункта меню на выходе программа выдает индекс найденного в массиве элемента или результат исследования.
\end{itemize}

\section{Выбор средств реализации}
Для реализации алгоритмов был выбран язык программирования \text{Python~\cite{python}}.

\section{Реализация алгоритмов}
\subsection{Алгоритм линейного поиска}
Реализация алгоритма линейного поиска элементов в массиве приведена на листинге~\ref{lst:linear_search}.
\lstinputlisting[label=lst:linear_search, caption=Алгоритм линейного поиска, basicstyle=\footnotesize, numbers=none]{lst/linear_search.py}

\subsection{Алгоритм бинарного поиска}
Реализация алгоритма бинарного поиска элементов в массиве приведена на листинге~\ref{lst:binary_search}.
\lstinputlisting[label=lst:binary_search, caption=Алгоритм бинарного поиска, basicstyle=\footnotesize, numbers=none]{lst/binary_search.py}

\section{Тестирование}
Для тестирования линейного и бинарного алгоритмов поиска элементов в массиве были составлены таблицы с входными данными (массив и искомый элемент), ожидаемым результатом (индексом) и полученным результатом от обоих способов.
\begin{table}[H]
	\centering
	\caption{Таблица тестов для алгоритмов поиска элементов в массиве}
	\begin{tabular}{|c|c|c|c|c|c|}
		\hline
		\multirow{2}{*}{Массив} & \multirow{2}{*}{Искомый элемент} & \multicolumn{2}{|c|}{Ожидание} & \multicolumn{2}{|c|}{Результат} \\ \cline{3-6}
		& & Линейный & Бинарный & Линейный & Бинарный \\ \hline
		[] & 1 & -1 & -1 & -1 & -1 \\ \hline
		[1, 2, 3] & 4 & -1 & -1 & -1 & -1 \\ \hline
		[1, 2, 3] & 1 & 1 & 1 & 1 & 1 \\ \hline
		[1, 2, 3] & 2 & 1 & 1 & 1 & 1 \\ \hline
		[1, 2, 3] & 3 & 2 & 2 & 2 & 2 \\ \hline
		[3, 2, 1] & 1 & 2 & 0 & 2 & 0 \\ \hline
	\end{tabular}
\end{table}

Для линейного и бинарного алгоритмов поиска элемента в массиве данные тесты были пройдены успешно.