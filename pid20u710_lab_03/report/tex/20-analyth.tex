\chapter{Аналитический раздел}
\section{Линейный алгоритм поиска}
Линейный (последовательный) поиск~--–~это метод поиска элемента в массиве или списке, при котором проверка осуществляется последовательно, начиная с первого элемента и продолжаясь до тех пор, пока не будет найден искомый элемент или не будет пройден весь набор данных~\cite{lin_search_alg}.

Линейный поиск имеет временную сложность $O(N)$, где $N$~--—~размер массива. В лучшем случае элемент находится сразу, что даёт сложность $O(1)$. В худшем случае требуется перебрать весь массив, либо элемент отсутствует, тогда сложность составляет $O(N)$. Алгоритм использует $O(1)$ дополнительной памяти, так как хранит лишь одну переменную для перебора элементов.

\section{Бинарный алгоритм поиска}
Бинарный (двоичный) поиск~---~это алгоритм поиска элемента в отсортированном массиве, основанный на принципе деления диапазона пополам. На каждом шаге алгоритм сравнивает искомый элемент с серединным значением массива и продолжает поиск в левой или правой половине, в зависимости от результата сравнения. Этот процесс повторяется, пока не будет найден нужный элемент или диапазон поиска не станет пустым~\cite{bin_search_alg}.  

Бинарный поиск имеет временную сложность $O(\log N)$. В лучшем случае, если искомый элемент находится сразу в середине алгоритм выполняется за $O(1)$. В худшем случаях количество проверок сокращается вдвое на каждом шаге, что приводит к временной сложности $O(\log N)$. Алгоритм требует $O(1)$ дополнительной памяти, но при рекурсивной реализации расходуется $O(\log N)$ памяти на хранение вызовов стека.