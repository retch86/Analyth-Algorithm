\chapter{Конструкторский раздел}
\section{Алгоритмы умножения матриц}
\label{section:shemas_algo}

\subsection{Стандартный алгоритм умножения матриц}
На рисунке~\ref{img:classic_multiply_mtrx} представлен алгоритм стандартного умножения матриц. 
\imgw{classic_multiply_mtrx.pdf}{Алгоритм стандартного умножения матриц}{img:classic_multiply_mtrx}{0.59\textwidth}

\newpage
\subsection{Алгоритм Винограда умножения матриц}
На рисунках~\ref{img:vinograd_1} и~\ref{img:vinograd_2} представлен алгоритм Винограда умножения матриц. 
\imgw{vinograd_1.pdf}{Алгоритм Винограда умножения матриц. Часть 1}{img:vinograd_1}{\textwidth}

\imgw{vinograd_2.pdf}{Алгоритм Винограда умножения матриц. Часть 2}{img:vinograd_2}{0.25\textwidth}

\newpage
\section{Модель оценки трудоемкости алгоритмов}
Введем модель оценки трудоемкости.

\begin{enumerate}
	\item Трудоемкость базовых операций.
	
	Пусть трудоемкость следующих операций равной 2: 
	$$
	\label{eq3:1}
	*, /, \, //, \%, \mathrel{*}=, \mathrel{/}=.
	$$
	
	Примем трудоемкость следующих операций равной 1:
	
	$$
	\label{eq3:2}
	=, +, -, +=, -=, ==, \mathrel{!}=, <, >, \leq, \geq, |, \&\&,, ||, [\ ], <<, >>.
	$$
	
	\item Трудоемкость цикла.
	
	Пусть трудоемкость цикла определяется по формуле (\ref{eq3:3}).
	
	\begin{equation}
		\label{eq3:3} 
		f = f_{init} + f_{comp} + N_{iter} * (f_{in} + f_{inc} + f_{comp}),
	\end{equation} 
	где:
	\begin{itemize}
		\item $f_{init}$: трудоемкость инициализации переменной-счетчика;
		\item $f_{comp}$: трудоемкость сравнения;
		\item $N_{iter}$: номер выполняемой итерации;
		\item $f_{in}$: трудоемкость команд из тела цикла;
		\item $f_{inc}$: трудоемкость инкремента;
		\item $f_{comp}$: трудоемкость сравнения.
	\end{itemize}
	\item Трудоемкость условного оператора. \\
	Пусть трудоемкость самого условного перехода равна 0 в лучшем случае, когда условие не выполняется, иначе~---~трудоемкости операций, относящихся к условному оператору $f_{if}$ (\ref{eq3:4}): 
	\begin{equation}
		\label{eq3:4}
		f_{if} = f_{comp\_if} +
		\begin{cases}
			0, & \text{лучший}\\
			f_{b}, & \text{худший}\\
		\end{cases}.
	\end{equation}
	
\end{enumerate}


\section{Вычисление трудоемкости алгоритмов}
Пусть во всех дальнейших вычислениях размер матрицы $A$ имеет $M \times N$, размер матрицы $B$ имеет $N \times Q$.

\subsection{Трудоемкость алгоритма умножения матриц в стандартном случае}
Трудоемкость $f_{stand}$ алгоритма стандартного умножения матриц вычисляется по формуле (\ref{eq3:5}):
\begin{multline}
	\label{eq3:5}
	f_{stand} = {\underset{=}{1}} + {\underset{<}{1}} + M({\underset{++}{2}} + {\underset{=}{1}} + {\underset{<}{1}} + Q({\underset{++}{2}} + {\underset{=}{1}} + {\underset{<}{1}} + N({\underset{++}{2}} + {\underset{[\ ]}{8}} + {\underset{*}{2}} + {\underset{=}{1}} + {\underset{+}{1}}))) = \\
	= 14MNQ + 4MQ + 4M + 2.
\end{multline}

\subsection{Трудоемкость алгоритма умножения матриц по Винограду}
Трудоемкость этого алгоритма состоит из следующих компонентов, определяемых по формуле (\ref{eq3:6}):
\begin{equation}
	\label{eq3:6}
	f_{vin} = f_{init} + f_{precomp} + f_{fill} + f_{even},
\end{equation}
где:
\begin{itemize}
	\item $f_{init}$~---~трудоемкость инициализации массивов для предварительного вычисления (\ref{eq3:7}):
	\begin{multline}
		\label{eq3:7}
		f_{init} = {\underset{=}{1}} + {\underset{<}{1}} + M({\underset{++}{2}} + {\underset{[\ ]}{1}} + {\underset{=}{1}}) + {\underset{=}{1}} + {\underset{<}{1}} + Q({\underset{++}{2}} + {\underset{[\ ]}{1}} + {\underset{=}{1}}) = \\
		= 2 + 4M + 2 + 4Q = 4 + 4M + 4Q; 
	\end{multline}
	
	
	\item $f_{precomp}$~---~трудоемкость предварительного заполнения строк матрицы А и столбцов матрицы B (\ref{eq3:8}):
	\begin{multline}
		\label{eq3:8}
		f_{precomp} = f_{rows} + f_{columns} = {\underset{=}{1}} + {\underset{<}{1}} + M({\underset{++}{2}} + {\underset{=}{1}} + {\underset{<}{1}} + \frac{n}{2}({\underset{++}{2}} + {\underset{[\ ]}{6}} + {\underset{=}{1}} + {\underset{+}{2}} + {\underset{*}{6}})) + \\
		+ {\underset{=}{1}} + {\underset{<}{1}} + Q({\underset{++}{2}} + {\underset{=}{1}} + {\underset{<}{1}} + \frac{N}{2}({\underset{++}{2}} + {\underset{[\ ]}{6}} + {\underset{=}{1}} + {\underset{+}{2}} + {\underset{*}{6}})) = \\
		= 2 + M(4 + \frac{N}{2} * 17) + 2 + Q(4 + \frac{N}{2} * 17) = \\
		= 4 + 4M + 4Q + \frac{17NM}{2} + \frac{17NQ}{2};
	\end{multline}
	
	
	\item $f_{even}$~---~трудоемкость заполнения результирующей матрицы (\ref{eq3:9}):
	\begin{multline}
		\label{eq3:9}
		f_{fill} = {\underset{=}{1}} + {\underset{<}{1}} + M({\underset{++}{2}} + {\underset{=}{1}} + {\underset{<}{1}} + Q({\underset{++}{2}} + {\underset{[\ ]}{4}} + {\underset{=}{1}} + {\underset{-}{2}} + {\underset{=}{1}} + {\underset{<}{1}} +\\
		+ \frac{N}{2}({\underset{++}{2}} + {\underset{[\ ]}{12}} + {\underset{=}{1}} +
		{\underset{+}{5}} + {\underset{*}{10}} + {\underset{/}{2}})) = 2 + M(4 + Q(11 + 16N)) = \\
		= 2 + 4M + 11MQ + 16MNQ;
	\end{multline}
	
	\item $f_{fill}$~---~трудоемкость для дополнения умножения в случае нечетной размерности матрицы. (\ref{eq3:10}):
	\begin{multline}
		\label{eq3:10}
		f_{fill} = {\underset{\%}{2}} + {\underset{==}{1}} + 
		\begin{cases}
			{\underset{=}{1}} + {\underset{<}{1}} + M({\underset{++}{2}} + {\underset{=}{1}} + {\underset{<}{1}} + Q({\underset{++}{2}} + {\underset{[\ ]}{8}} + \\
			+ {\underset{=}{1}} + {\underset{+}{1}} + {\underset{-}{2}})),\\
			0\\
		\end{cases} = \\
		= 3 + \begin{cases}
			2 + 4M + 14MQ,\\
			0\\
		\end{cases}.
	\end{multline}
	
	Результирующая трудоемкость алгоритма Винограда составляет \newline $f_{vin} \approx 16MNQ$.
\end{itemize}
\subsection{Трудоемкость оптимизированного алгоритма умножения матриц по Винограду}
Трудоемкость этого алгоритма определяется из следующих компонентов по формуле (\ref{eq3:11}):
\begin{equation}
	f_{optim} = f_{init} + f_{precomp} + f_{fill}.
	\label{eq3:11}
\end{equation}
где:
\begin{itemize}
	\item $f_{init}$~---~определяется по формуле (\ref{eq3:7}) в добавок с другими компонентами (\ref{eq3:12});
	\begin{equation}
		f_{init} = 4 + 4M + 4Q + {\underset{=}{2}} + {\underset{\%}{2}} + {\underset{-}{1}} + {\underset{==}{1}} + 
		\begin{cases}
			{\underset{-=}{1}},\\
			0
		\end{cases} = 7 + 4M + 4Q + \begin{cases}
			{\underset{-=}{1}},\\
			0
		\end{cases}.
		\label{eq3:12}
	\end{equation}
	
	\item $f_{precomp}$~---~трудоемкость предварительного заполнения строк матрицы $А$ и столбцов матрицы $B$ (\ref{eq3:13}):
	\begin{multline}
		f_{precomp} = {\underset{=}{1}} + {\underset{<}{1}} + M({\underset{++}{2}} + {\underset{=}{2}} + {\underset{<}{1}} + N({\underset{+=}{1}} + {\underset{<<}{1}} + {\underset{+}{1}} + {\underset{[\ ]}{4}}) + {\underset{[\ ]}{1}} + {\underset{=}{1}}) + \\
		+ {\underset{=}{1}} + {\underset{<}{1}} + Q({\underset{++}{2}} + {\underset{=}{2}} + {\underset{<}{1}} + N({\underset{+=}{1}} + {\underset{[\ ]}{4}} + {\underset{<<}{1}} + {\underset{+}{1}}) + {\underset{=}{1}}) = \\
		= 2 + 7M + 7MN + 2 + 5Q + 8NQ = 9MN + 9NQ + 7M + 5Q + 4;
		\label{eq3:13}
	\end{multline}
	
	\item $f_{fill}$~---~трудоемкость для заполнения матрицы (\ref{eq3:14}):
	\begin{multline}
		f_{fill} = {\underset{=}{1}} + {\underset{<}{1}} + M({\underset{++}{2}} + {\underset{=}{2}} + {\underset{<}{1}} + Q({\underset{++}{2}} + {\underset{=}{2}} + {\underset{-}{1}} + {\underset{+}{1}} + {\underset{[\ ]}{2}} + {\underset{<}{1}} + \\
		+ \frac{N}{2}({\underset{+=}{2}} + {\underset{[\ ]}{8}} + {\underset{+}{4}} + {\underset{<<}{1}}) + {\underset{==}{1}} 
		+  \begin{cases}
			{\underset{+=}{1}} + {\underset{[ \ ]}{4}} + {\underset{<<}{1}}\\
			0
		\end{cases} + {\underset{[\ ]}{2}} + {\underset{=}{1}})) = \\
		= 2 + M(4 + Q(\frac{15}{2}) + 13 + \begin{cases}
			6\\
			0
		\end{cases})) = \\
		8.5MNQ + 14MQ + 4M + 2 + \begin{cases}
			6\\
			0
		\end{cases} MQ;
		\label{eq3:14}
	\end{multline}
	Результирующая трудоемкость оптимизированного алгоритма Винограда для лучшего и худшего случая составляет (\ref{eq3:15}):
	\begin{equation}
		f_{fill} \approx 8.5MNQ.
		\label{eq3:15}
	\end{equation}
	
\end{itemize}

\section*{Вывод}
В данном разделе были приведены схемы алгоритмов умножения матриц стандартным образом, Винограда, проведена теоретическая оценка трудоемкости алгоритмов. Стандартный алгоритм умножения матриц имеет трудоемкость $14MNQ$, стандартный по Винограду $16MNQ$, оптимизированный по Винограду $8.5MNQ$.
