\chapter{Аналитический раздел}
\section{Определение матрицы}
Матрицей размера $m \times n$ называется прямоугольная таблица элементов некоторого множества 
(например, чисел или функций), имеющая $m$ строк и $n$ столбцов~\cite{angem}.
Элементы $a_{ij}$ , из которых составлена матрица, называются элементами матрицы.
Условимся, что первый индекс $i$ элемента $a_{ij}$
соответствует номеру строки, второй индекс $j$~–~номеру столбца, в котором расположен элемент $a_{ij}$.
Матрица $A$ может быть записана по формуле (\ref{eq:ref1}):

\begin{equation}
	A = \left(
	\begin{array}{cccc}
		a_{11} & a_{12} & \ldots & a_{1n} \\
		a_{21} & a_{22} & \ldots & a_{2n} \\
		\vdots & \vdots & \ddots & \vdots \\
		a_{m1} & a_{m2} & \ldots & a_{mn}
	\end{array}
	\right).
	\label{eq:ref1}
\end{equation}

\section{Стандартный алгоритм умножения матриц}
Произведением матрицы $A = (a_{ij})$, имеющей $m$ строк и $n$ столбцов, на матрицу $B = (b_{ij})$, имеющую
$n$ строк и $p$ столбцов, называется матрица $c_{ij}$, имеющая $m$ строк и $p$ столбцов, у которой элемент 
$C = (c_{ij})$ определяется по формуле (\ref{eq:ref2}):

\begin{equation}
	\begin{array}{cc}
		c_{ij} = \sum\limits_{r=1}^m a_{ir}b_{ri} & (i=1,2,\dots n; j=1,2,\dots p)
	\end{array}.
	\label{eq:ref2}
\end{equation}

\section{Алгоритм умножения матриц по Винограду}
Обозначим $i$ строку матрицы $А$ как  $\overline{u}$, $j$ столбец матрицы $В$ как $\overline{v}$~\cite{winograd-origin}. Тогда элемент $c_{ij}$ определяется по формуле (\ref{eq:ref3}):

\begin{equation}
	c_{ij} = \overline{u} \times \overline{v} =
	\begin{pmatrix} u_{1} & u_{2} & u_{3} & u_{4}\end{pmatrix}
	\begin{pmatrix} v_{1} \\ v_{2} \\ v_{3} \\ v_{4}\end{pmatrix} =
	u_{1}v_{1} + u_{2}v_{2} + u_{3}v_{3} + u_{4}v_{4}.
	\label{eq:ref3}
\end{equation}

Эту формулу можно представить в следующем виде (\ref{eq:ref4}):
\begin{equation}
	(u_{1}+v_{2})(u_{2}+v_{1}) + (u_{3}+v_{4})(u_{4}+v_{3}) - u_{1}u_{2} - u_{3}u_{4} - v_{1}v_{2} - v_{3}v_{4}
	\label{eq:ref4}
\end{equation}

На первый взгляд кажется, что выражение (\ref{eq:ref4}) задает больше работы, чем первое: вместо четырех умножений насчитывается их шесть, а вместо трех сложений~--- десять. Выражение в правой части формулы  можно вычислить заранее и затем повторно использовать. На практике это означает, что над предварительно обработанными элементами придется выполнять лишь первые два умножения и последующие пять сложений, а также дополнительно два сложения.