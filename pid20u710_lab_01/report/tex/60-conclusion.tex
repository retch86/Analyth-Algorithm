\part*{ЗАКЛЮЧЕНИЕ}
\addcontentsline{toc}{part}{\textbf{ЗАКЛЮЧЕНИЕ}}

В ходе выполнения лабораторной работы были исследованы алгоритмы нахождения расстояния Левенштейна и Дамерау~---~Левенштейна. Были выполнены описание каждого из этих алгоритмов, приведены соответствующие математические расчёты.  

При тестировании каждого их них и анализе временных характеристик и объема потребляемой памяти сделаны следующие выводы: выбор алгоритма Дамерау~---~Левенштейна является оптимальным решением ввиду того, что чаще всего необходимо исправлять ошибки, связанные с обменом двух соседних символов. В ином случае этот алгоритм является проигрышным как по времени, так и по памяти в сравнении с различными реализациями алгоритма Левенштейна. Рекурсивный алгоритм Левенштейна с кэшем в виде матрицы выигрывает по скорости выполнения у данной группы алгоритмов, но он проигрывает по использованию памяти за счет большего числа вызовов. Таким образом, в ситуациях, не связанных с транспозицией, следует использовать итеративный алгоритм.