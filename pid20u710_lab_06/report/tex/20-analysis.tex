\chapter{Аналитическая часть}
В этом разделе будет описана задача коммивояжера и методы ее решения: метод полного перебора и на основе муравьиного алгоритма.

\section{Задача коммивояжера}
В этой задаче рассматривается $n$ городов и матрица попарных расстояний между ними. 
Необходимо найти такую последовательность посещения городов, чтобы пройденное расстояние было как можно меньше.
Каждый город должен быть посещен ровно один раз и коммивояжеру необходимо оказаться в городе, из которого он начал свое движение~\cite{salesman}. 

\section{Методы решения задачи коммивояжера}
\textbf{Полный перебор.}
Суть этого решения заключается в переборе всех возможных вариантов замкнутых путей и в выборе кратчайшего из них. 

\textbf{Муравьиный алгоритм.}
Этот метод основан на принципах, описывающих поведение колонии муравьев.
Они используют феромоны для общения друг с другом и поиска пути к пище.
Если путь длинный, феромон испаряется, и последующая пара выбирает другой путь с большим количеством феромона, оставляя наибольшее его количество на кратчайшем пути~\cite{shtovba}.

Введем целевую функцию~(\ref{eq:d_func}), которая описывает привлекательность ребра
\begin{equation}
	\label{eq:d_func}
	\eta_{ij} = 1 / D_{ij},
\end{equation}
где $D_{ij}$~---~расстояние от текущего пункта $i$ до заданного пункта $j$.

Вероятность перехода из пункта $i$ в пункт $j$ определяется по формуле~(\ref{eq:prob}):
\begin{equation}
	\label{eq:prob}
	P_{i,j}={\frac {(\tau_{i,j}^{\alpha})(\eta_{i,j}^{\beta })}{\sum (\tau_{i,j}^{\alpha})(\eta_{i,j}^{\beta})}},
\end{equation}
где:
\begin{itemize}
	\item $\tau_{i,j}$~---~количество феромонов на ребре от $i$ до $j$;
	\item $\eta_{i,j}$~---~привлекательность пути от $i$ до $j$;
	\item $\alpha$~---~параметр влияния расстояния;
	\item $\beta$~---~параметр влияния феромона.
\end{itemize}

В случае $\alpha = 0$ выбирается ближайший город и алгоритм становится <<жадным>>, то есть выбираются только оптимальные или самые короткие расстояния.

Если $\beta = 0$, то работает лишь усиление феромонами, что влечет за собой сужение пространства поиска оптимального решения~\cite{shtovba}.

После происходит обновление феромона на пройденных путях по формуле~(\ref{eq:upd_ph}), в случае, если $p$~---~коэффициент испарения феромона, $N$~---~количество феромонов, $Q$~---~некоторая константа порядка длины путей, $L_{k}$~---~длина пути муравья с номером $k$~\cite{shtovba}.
\begin{equation}\label{eq:upd_ph}
	\tau_{ij}(t+1) = (1-p)\tau_{ij}(t) + \Delta \tau_{ij},~~\Delta \tau_{ij} =
	\displaystyle\sum_{k=1}^N \tau^k_{ij},
\end{equation}
где
\begin{equation}\label{eq:3}
	\Delta \tau^k_{ij} = \begin{cases}
		\frac{Q}{L_k}, & \quad \textrm{ребро посещено $k$-ым муравьем,} \\
		0, & \quad \textrm{иначе.}
	\end{cases}
\end{equation}

Так как вероятность (\ref{eq:prob}) перехода в заданную точку не должна быть равна нулю, для этого нужно обеспечить неравенство $\tau_{ij} (t)$ нулю путем введения дополнительного минимально возможного значения феромона $\tau_{min}$.
В случае, если $\tau_{ij} (t+1)$ принимает значение, меньшее $\tau_{min}$, откатывать феромон до этой величины.

Одной из модификаций муравьиного алгоритма является элитарная муравьиная система.
При таком подходе искусственно вводятся <<элитные>> муравьи, усиливающие уровень феромонов, оптимального на данный момент маршрута.

\section*{Вывод}
В этом разделе была рассмотрена задача коммивояжера и способы ее решения~---~полным перебором и муравьиным алгоритмом.
