\chapter{Исследовательская часть}

\section{Технические характеристики}
Технические характеристики устройства, на котором выполнялось тестирование:
\begin{itemize}
	\item операционная система: Windows 10 Pro;
	\item память: 8 Гб;
	\item процессор: Intel(R) Core(TM) i5-8265U CPU @ 1.60 ГГц 1.80 ГГц;
	\item 12 логических ядра.
\end{itemize}
Тестирование проводилось на ноутбуке, который был подключен к сети питания. 
Во время проведения тестирования ноутбук был нагружен только встроенными приложениями окружения, самим окружением и системой тестирования.

\newpage

\section{Демонстрация работы программы}
На рисунках~\ref{img:prog_01}~--~\ref{img:prog_03} продемонстрирована работа программы.
\includeimage
	{prog_01}
	{f}
	{h} 
	{.4\textwidth} 
	{Демонстрация работы программы (меню и исходная матрица)} 

\includeimage
	{prog_02}
	{f}
	{h} 
	{.5\textwidth} 
	{Результат работы алгоритма полного перебора} 

\newpage

\includeimage
	{prog_03}
	{f}
	{h} 
	{.5\textwidth} 
	{Результат работы муравьиного алгоритма} 
	

\section{Параметризация муравьиного алгоритма}
Параметризация была проведена на трех классах данных: \ref{eq:kd1}, \ref{eq:kd2} и \ref{eq:kd3}.
Таблицы имеют различный разброс расстояний: 1000, 100, 10 соответственно.
Алгоритм был запущен для набора значение $\alpha, eva \in (0, 1)$.

Результирующая таблица значений параметризации будет состоять из следующих столбцов:
\begin{enumerate}
	\item $\alpha$~---~параметр $\alpha$ при вычислении вероятности перехода в новый город;
	\item $eva$~---~коэффициент испарения;
	\item $days$~---~количество дней жизни колонии;
	\item $optim$~---~результат решения полным перебором;
	\item $delta$~---~разность между решением полным перебором и решением муравьиного алгоритма.
\end{enumerate}
Замеры проводились 10 раз и выбирался результат с максимальным $delta$ от результата перебора.
Для полученных таблиц~\ref{t:params_1}--\ref{t:params_3} соответственно использовались матрицы расстояний~(\ref{eq:kd1}--\ref{eq:kd3}).

\begin{equation}
	\label{eq:kd1}
	K_{1} =\begin{bmatrix}
	 0 & 2525 & 934 & 3096 & 6574 & 3486 & 1970 & 1328 & 2672 & 2547 \\
		2525 & 0 & 4315 & 4726 & 3236 & 699 & 5121 & 2290 & 6050 & 5950 \\
		934 & 4315 & 0 & 3569 & 5892 & 4410 & 1161 & 2084 & 1997 & 1831 \\
		3096 & 4726 & 3569 & 0 & 7728 & 4259 & 2888 & 3917 & 3239 & 3330 \\
		6574 & 3236 & 5892 & 7728 & 0 & 3804 & 7021 & 4099 & 7889 & 7717 \\
		3486 & 699 & 4410 & 4259 & 3804 & 0 & 5069 & 2539 & 5974 & 5899 \\
		1740 & 5121 & 1161 & 2888 & 7021 & 5069 & 0 & 3061 & 1010 & 831 \\
		1328 & 2290 & 2084 & 3917 & 4099 & 2539 & 3061 & 0 & 3987 & 3851 \\
		2672 & 6050 & 1997 & 3239 & 7889 & 5974 & 1010 & 3987 & 0 & 205 \\
		2547 & 5950 & 1831 & 3330 & 7717 & 5899 & 831 & 3851 & 205 & 0 \\
	\end{bmatrix}
\end{equation}

\begin{equation}
	\label{eq:kd2}
K_{2} = \begin{bmatrix}
	0 & 50 & 72 & 36 & 34 & 89 & 85 & 8 & 22 & 11 \\
	50 & 0 & 33 & 23 & 78 & 61 & 16 & 1 & 92 & 31 \\
	72 & 33 & 0 & 100 & 51 & 9 & 66 & 58 & 36 & 23 \\
	36 & 23 & 100 & 0 & 41 & 43 & 45 & 51 & 31 & 67 \\
	34 & 78 & 51 & 41 & 0 & 33 & 22 & 38 & 32 & 63 \\
	89 & 61 & 9 & 43 & 33 & 0 & 46 & 71 & 9 & 32 \\
	85 & 16 & 66 & 45 & 22 & 46 & 0 & 41 & 51 & 78 \\
	8 & 1 & 58 & 51 & 38 & 71 & 41 & 0 & 100 & 23 \\
	22 & 92 & 36 & 31 & 32 & 9 & 51 & 100 & 0 & 33 \\
	11 & 31 & 23 & 67 & 63 & 32 & 78 & 23 & 33 & 0 \\
\end{bmatrix}
\end{equation}

\begin{equation}
	\label{eq:kd3}
K_{3} = \begin{bmatrix}
	0 & 1 & 9 & 0 & 9 & 4 & 2 & 10 & 7 & 10 \\
	1 & 0 & 9 & 10 & 9 & 1 & 9 & 1 & 9 & 8 \\
	9 & 9 & 0 & 4 & 8 & 7 & 4 & 6 & 8 & 5 \\
	0 & 10 & 4 & 0 & 9 & 0 & 1 & 7 & 3 & 2 \\
	9 & 9 & 8 & 9 & 0 & 8 & 9 & 8 & 7 & 5 \\
	5 & 1 & 7 & 0 & 8 & 0 & 1 & 4 & 8 & 2 \\
	2 & 9 & 4 & 1 & 9 & 1 & 0 & 9 & 8 & 9 \\
	10 & 1 & 6 & 7 & 8 & 4 & 9 & 0 & 4 & 10 \\
	7 & 9 & 8 & 3 & 7 & 8 & 8 & 4 & 0 & 9 \\
	10 & 8 & 5 & 2 & 4 & 2 & 9 & 10 & 9 & 0 \\
\end{bmatrix}
\end{equation}

\section{Временные характеристики}
Сравнение алгоритмов по времени выполнения производилось при изменении количества городов $n$ от 2 до 10 с шагом 1.
В результате замеров времени была получена таблица~\ref{t:timings}.
Замеры проводились 10 раз, затем бралось их среднее арифметическое значение. 
По таблице~\ref{t:timings} был построен график~\ref{img:time_plot}.

\begin{table}[ht]
	\centering
	\caption{Результаты измерений реализаций алгоритмов при изменении количества городов}
	\begin{tabular}{|c|c|c|c|}
	\hline
	n & Алгоритм полного перебора (с) & Муравьиный алгоритм (с) \\ \hline
   2 &   0.000059 &   0.002753 \\ \hline
3 &   0.000045 &   0.012209 \\ \hline
4 &   0.000092 &   0.027333 \\ \hline
5 &   0.000366 &   0.060201 \\ \hline
6 &   0.001566 &   0.101122 \\ \hline
7 &   0.011576 &   0.189127 \\ \hline
8 &   0.099957 &   0.325559 \\ \hline
9 &   1.164084 &   0.576964 \\ \hline
10 &  13.032171 &   0.655146 \\ \hline

	\end{tabular}
\label{t:timings}
\end{table}

\includeimage
	{time_plot} 
	{f} 
	{H} 
	{1\textwidth} 
	{Сравнение реализаций алгоритмов по времени работы при изсенении размера матрицы смежности}

\section*{Выводы}
С увлечение числа дней жизни колонии $delta$ уменьшается, вне зависимости от разброса расстояний. 
От значений $\alpha$ и $eva$ зависимостей не обнаружено, следовательно, значения параметров должны устанавливаться для каждой конкретной задачи отдельно. 

Из таблицы~\ref{t:timings} выяснено: при малых размерах матриц реализация алгоритма полного перебора тратит меньше времени на получение результата.

При размерах матрицы больше 9 реализация алгоритма полного перебора работает в 1.45 раз больше, чем реализация муравьиного алгоритма.
Таким образом, выбор решения задачи определяется размером матрицы расстояний между городами.
