\chapter{Конструкторская часть}
В этом разделе будут рассмотрены схемы алгоритма полного перебора и муравьиного алгоритма.

\section{Схемы алгоритмов}
На рисунке~\ref{img:combAlg} представлен алгоритм перебора всех возможных путей.

\includeimage
	{combAlg} 
	{f}
	{h}
	{.6\textwidth}
	{Схема алгоритма полного перебора}

\clearpage

На рисунке~\ref{img:antAlg} представлен муравьиный алгоритм.

\includeimage
	{antAlg}
	{f}
	{h} 
	{.76\textwidth} 
	{Схема муравьиного алгоритма} 

\clearpage

\section{Оценка трудоемкости}
Задача коммивояжера считается $NP$~-~трудной.
Сложность алгоритма полного перебора~---~$O(n!)$~\cite{salesman}.

Сложность муравьиного алгоритма~---~$O(t_{max} \cdot m \cdot n^2)$.
Она зависит от времени жизни колонии, количества городов и количества муравьев в колонии~\cite{ulyanov}. 

Поскольку в разработанной реализации количество муравьев и городов одинаково, то трудоемкость муравьиного алгоритма равна $O(t_{max}~\cdot~n^3)$.

\section*{Вывод}
В этом разделе были построены схемы описанных алгоритмов.
