\chapter*{\hfill{\centering  ЗАКЛЮЧЕНИЕ}\hfill}
\addcontentsline{toc}{chapter}{ЗАКЛЮЧЕНИЕ}
Графовые модели предоставляют мощные инструменты для анализа программного кода, которые помогают структурировать информацию, выявлять взаимосвязи и зависимости, а также находить проблемы и уязвимости. 
Их применимость охватывает множество аспектов, включая статический и динамический анализ, оптимизацию производительности, упрощение структуры кода и понимание изменений состояния. 
С учетом всё более сложных и больших кодовых баз, графовые модели становятся незаменимыми в области программной инженерии и анализа кода.

Цель данной лабораторной работы была достигнута.

Также были решены следующие задачи:
\begin{itemize}
	\item даны определения графовых моделей;
	\item выделен законченный фрагмент кода на 15+ значащих строк кода;
	\item выполнено построение 4 графов;
	\item сделан вывод о применимости графовых моделей к задаче анализа программного кода.
\end{itemize}