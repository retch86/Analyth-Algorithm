\chapter{Аналитический раздел}
В данном разделе представлено описание сути конвертерной обработки данных и используемый алгоритм обработки матрицы для конвейерных вычислений.

\section{Конвейерная обработка данных}
Конвейерная обработка данных — способ организации вычислений, используемый в современных процессорах и контроллерах с целью повышения их производительности за счет увеличения числа инструкций, выполняемых в единицу времени — эксплуатация параллелизма на уровне инструкций \cite{conveyor}.

Конвейерную обработку используют для совмещения этапов выполнения разных команд. Производительность возрастает благодаря тому, что одновременно на различных ступенях конвейера выполняются несколько команд. Такая обработка данных в общем случае основана на разделении функции на более мелкие части, называемые лентами, и выделении для каждой из них отдельного блока аппаратуры. Таким образом, обработку любой машинной команды можно разделить на несколько этапов (лент), организовав передачу данных от одного этапа к следующему.

Конвейеризация позволяет увеличить пропускную способность процессора (количество команд, завершающихся в единицу времени), но она не сокращает время выполнения отдельной команды. 
В действительности она даже несколько увеличивает время выполнения каждой команды из-за накладных расходов, связанных с хранением промежуточных результатов. 
Однако увеличение пропускной способности означает, что программа будет выполняться быстрее по сравнению с простой, не конвейерной схемой.

\section{Описание алгоритмов конвейерной обработки}
В данной лабораторной работе на основе конвейерной обработке данных будет обрабатываться матрица. В качестве алгоритмов на каждую из трех лент были выбраны следующие действия.

\begin{enumerate}
	\item Нахождение наименьшего элемента в матрице.
	\item Запись в каждую ячейку матрицы остатка от деления текущего элемента на минимальный элемент.
	\item Нахождение суммы элементов полученной матрицы.
\end{enumerate}

\section*{Вывод}

В разделе было рассмотрено понятие конвейрной обработки данных, а также описаны алгоритмы для обработки матрицы на каждой из трех лент конвейера.