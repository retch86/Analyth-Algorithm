\chapter{Конструкторский раздел}
В данном разделе приводятся схемы последовательного и параллельного алгоритмов сортировки слиянием, а также обоснование необходимости объединения промежуточных значений.

\section{Алгоритмы сортировки слиянием}
\subsection{Последовательная версия}
Алгоритм последовательной сортировки слиянием представлен на рисунке \ref{img:merge_sort_algo}.
\imgw{merge_sort_algo.pdf}{Алгоритм последовательной сортировки слиянием}{img:merge_sort_algo}{0.65\textwidth}
\newpage
Алгоритм слияния результатов представлен на рисунке \ref{img:merge_algo}.
\imgw{merge_algo.pdf}{Алгоритм слияния результатов}{img:merge_algo}{0.65\textwidth}
\subsection{Параллельная версия}
Алгоритм параллельной сортировки слиянием представлен на \text{рисунке \ref{img:merge_sort_parall_algo}}.
\imgw{merge_sort_parall_algo.pdf}{Алгоритм параллельной сортировки слиянием}{img:merge_sort_parall_algo}{0.70\textwidth}
Главный поток выполняет создание вспомогательных потоков, назначение каждому из нему задания и запуск этого потока. После выполняется шаг объединения промежуточных значений, которые затем сливаются в единый отсортированный массив главным потоком. Средства синхронизации не требуются.  
\newpage
Алгоритм работы вспомогательного потока представлен на рисунке \ref{img:add_thread_algo}.
\imgw{add_thread_algo.pdf}{Алгоритм работы вспомогательного потока}{img:add_thread_algo}{0.65\textwidth}

\section*{Вывод}
В данном разделе были приведены схемы последовательного и параллельного алгоритмов сортировки слиянием. Средства синхронизации не требуются, необходимо использовать объединение результатов вспомогательных потоков.
