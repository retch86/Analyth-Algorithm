\part*{ВВЕДЕНИЕ}
\addcontentsline{toc}{part}{\textbf{ВВЕДЕНИЕ}}

При необходимости максимально эффективно использовать ресурсы системы для выполнения множества задач, важно стремиться к увеличению скорости работы программ. Недоступность возможности увеличения тактовой частоты процессора по определённым техническим причинам не означает, что нет альтернативного способа повышения производительности. Один из таких способов заключается в использовании многоядерных процессоров, что требует особого подхода к программированию.

Параллельное программирование --  подход в технологии разработки программного обеспечения, которые основывается на потоках \cite{parallel}. Поток -- часть кода программы, которая может выполняться параллельно с другими частями кода программы. Многопоточность -- способность центрального процессора или одного ядра в многоядерном процессоре одновременно выполнять несколько потоков.

Целью лаборатоной работы является изучение и реализация параллельного программирования для сортировки массива методом слияния. Для её достижения необходимо выполнить следующие задачи:
\begin{itemize}
	\item рассмотреть случай последовательного и параллельного алгоритма сортировки слиянием;
	\item привести схемы алгоритмов последовательной и параллельной сортировки слиянием;
	\item реализовать приведенные алгоритмы;
	\item выполнить тестирование реализации алгоритмов методом черного ящика;
	\item провести сравнительный анализ этих алгоритмов по выполнению времени на основе экспериментальных данных.
\end{itemize}